\documentclass[11pt, fleqn, a4paper]{report}
%a4paper : 21.0cm * 29.7cm

\usepackage{xeCJK}
\setCJKmainfont[BoldFont=STFangsong, ItalicFont=STKaiti]{STSong}
\setCJKsansfont[BoldFont=STHeiti]{STSong}
\setCJKmonofont{STFangsong}

\usepackage{amsmath}
\usepackage{amssymb,amsfonts}
\usepackage{tabularx}
%\usepackage{longtable}
\usepackage{graphicx}
\usepackage{multirow}
\usepackage{tikz}
\usepackage[T1]{fontenc}
\usepackage{upquote}
\usepackage[colorlinks, linkcolor=blue, anchorcolor=blue, citecolor=blue, urlcolor=blue]{hyperref}
\usepackage{ltxtable, filecontents}

\setlength{\topmargin}{0cm}

\setlength{\oddsidemargin}{0.4cm}
\setlength{\evensidemargin}{0.4cm}
\setlength{\hoffset}{-0.2in}
\setlength{\textwidth}{440pt}
\setlength{\textheight}{650pt}
\setlength{\parindent}{0pt}
\setlength{\parskip}{5pt}

\setlength{\mathindent}{0pt}

\usepackage{color}
\usepackage{xcolor}
\usepackage{listings}

\usepackage{caption}
\DeclareCaptionFont{white}{\color{white}}
\DeclareCaptionFormat{listing}{\colorbox{lightgray}{\parbox{\textwidth}{#1#2#3}}}
\captionsetup[lstlisting]{format=listing,labelfont=white,textfont=white}

\definecolor{mygreen}{rgb}{0,0.6,0}
\definecolor{mygray}{rgb}{0.5,0.5,0.5}
\definecolor{mymauve}{rgb}{0.58,0,0.82}

\lstset{
	basicstyle=\footnotesize,
	breaklines=true,
	commentstyle=\color{mygreen},
	numbers=left,
	numbersep=5pt,
	numberstyle=\tiny\color{mygray},
	stringstyle=\color{mymauve},
	showstringspaces=false,
	showspaces=false,
	showtabs=false,
	tabsize=2,
	framexleftmargin=10mm,
	frame=none,
	backgroundcolor=\color[RGB]{245,245,244},
	keywordstyle=\bf\color{blue},
	identifierstyle=\bf,
	numberstyle=\color[RGB]{0,192,192},
	commentstyle=\it\color[RGB]{0,96,96},
	stringstyle=\rmfamily\slshape\color[RGB]{128,0,0}
}

\newcommand{\tabincell}[2]{\begin{tabular}{@{}#1@{}}#2\end{tabular}}

\usepackage{algorithm}
\usepackage{algorithmicx}
\usepackage{algpseudocode}


\begin{document}

\begin{titlepage}
\vspace*{40mm}
\begin{center}
{\Huge Homework \quad 1}\\[30mm]

{\Large 5130309059 \quad \quad 李佳骏}\\[3mm]
\texttt{taringlee@sjtu.edu.cn}\\[10mm]

2015.9.21

\end{center}
\end{titlepage}

\section*{Exercise 1.8}
这题我最初的想法是:

在A中找一个她的clique中度最小的点x,在B中找一个他的IS(independent set)中度最大的点y。然后把不和x有边的点、和y没有边的点都从图里删掉。这样只需要保证每次删掉至少一半点就可以解决这个问题。

当然这并不能被保证。没有想到$\frac{1}{2}$的那个构造,所以看了答案,以下是根据自己的理解写下的标准答案:

我们通过构造一个合理的协议(Protocol)来解决这个问题。在这个协议中Alice和Bob将共同维护一个集合$V'$,表示答案$C \cap I$所在的可能集合。初始化为$V' = V$。对其进行若干次折半操作。每次折半操作将至少减少$V'$一半的元素个数。这样至多进行$\frac{n}{2}$次折半操作即可得到解(其中$n=|V|$)。

若每一次折半操作的通讯传输是$\mathrm{O}(\log_2{n})$bits。这整体的通讯复杂度$D(CIS_G)= \mathrm{O}(\log_2^2{n})$。

定义$n(x; Y)$表示点集Y中所有和x相邻的点的集合。显然的,它是Y的子集。
以下是每一次折半操作的协议设计:
\begin{itemize}
\item Alice寻找一个点$x \in C \cap V'$,且$n(x; V') \leq \frac{|V'|}{2}$。如若找不到这样的点,向Bob传输“0”(1 bit);否则向Bob传输“1”和这个点的标号x($1 + \log{n}$bits)。
\item 若Bob收到的信息为“1”和x,Bob将优先判断$x \in I$是否成立。若成立则输出答案x。否则执行减半操作$V' = n(x; V')$,并告知Alice(例如传输“0”),该次操作结束。若Bob收到的信息为“0”。则Bob寻找一个点$y \in I \cap V'$,且$n(x; V') \geq \frac{|V'|}{2}$。如若找不到这个点,将输出"0"表示最终结果($C \cap I$为空);否则传输“1”和这个点的标号y($1 + \log{n}$bits)给Alice。
\item 当Alice收到“1”和y时,Alice将优先判断$y \in C$是否成立。若成立则立即输出答案y。否则执行减半操作$V' = V' - n(y; V')$,并告知Bob(例如传输“0”),该次操作结束。
\end{itemize}

通过上述操作,不难发现。如果答案$v'$存在,无论怎么进行操作,始终满足$v' \in V'$。且$x$与$y$中有一个肯定可以选择为$v'$。而由于选点的不等式限制的缘故,每次操作将一定至少是折半的操作。所以该操作满足上述要求,$D(CIS_G)= \mathrm{O}(\log_2^2{n})$得证。

\newpage

\section*{Exercise 1.31}
Let $f: X \times Y \rightarrow (0,1)$ be a Boolean function.

\subsection*{Subproblem 1}
Prove that if $f$ is such that all the rows of $M_f$ are distinct, then $D(f) \geq \log{\log{|X|}}$.

在当前条件下,证明$rank(f) \geq \log{|X|}$。其中$rank(f)$指的是矩阵$M_f$的秩(线性秩)。

若$|X|$行都不相同,则至少需要$\log{|X|}$列以表示$2^{\log{|X|}} = |X|$组解,所以列秩至少为$\log{|X|}$。

由矩阵秩等于列秩可得,$rank(f) \geq \log{|X|}$。

根据定理$$D(f) \geq \log{rank(f)}$$可推导得$$D(f) \geq \log{\log{|X|}}$$

\subsection*{Subproblem 2}
Prove that $D(f) \leq rank(f) + 1$.

通过构造协议进行证明。

Alice和Bob使用相同的算法计算得到矩阵$M_f$的一组相同的基向量。Alice传输$rank(f)$个bits以表示这个基向量的一组构造。而Bob只需要通过这个构造计算得出这个解,就能得到Alice的信息。且以此推出答案并输出(需要一个bit)。

因此$D(f) \leq rank(f) + 1$成立。

\end{document}