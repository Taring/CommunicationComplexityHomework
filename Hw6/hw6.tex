\documentclass[11pt, fleqn, a4paper]{report}
%a4paper : 21.0cm * 29.7cm

\usepackage{xeCJK}
\setCJKmainfont[BoldFont=STFangsong, ItalicFont=STKaiti]{STSong}
\setCJKsansfont[BoldFont=STHeiti]{STSong}
\setCJKmonofont{STFangsong}

\usepackage{amsmath}
\usepackage{amssymb,amsfonts}
\usepackage{tabularx}
%\usepackage{longtable}
\usepackage{graphicx}
\usepackage{multirow}
\usepackage{tikz}
\usepackage[T1]{fontenc}
\usepackage{upquote}
\usepackage[colorlinks, linkcolor=blue, anchorcolor=blue, citecolor=blue, urlcolor=blue]{hyperref}
\usepackage{ltxtable, filecontents}
\usepackage{mathrsfs}  

\setlength{\topmargin}{0cm}

\setlength{\oddsidemargin}{0.4cm}
\setlength{\evensidemargin}{0.4cm}
\setlength{\hoffset}{-0.2in}
\setlength{\textwidth}{440pt}
\setlength{\textheight}{650pt}
\setlength{\parindent}{0pt}
\setlength{\parskip}{5pt}

\setlength{\mathindent}{0pt}

\usepackage{color}
\usepackage{xcolor}
\usepackage{listings}

\usepackage{caption}
\DeclareCaptionFont{white}{\color{white}}
\DeclareCaptionFormat{listing}{\colorbox{lightgray}{\parbox{\textwidth}{#1#2#3}}}
\captionsetup[lstlisting]{format=listing,labelfont=white,textfont=white}

\definecolor{mygreen}{rgb}{0,0.6,0}
\definecolor{mygray}{rgb}{0.5,0.5,0.5}
\definecolor{mymauve}{rgb}{0.58,0,0.82}

\lstset{
	basicstyle=\footnotesize,
	breaklines=true,
	commentstyle=\color{mygreen},
	numbers=left,
	numbersep=5pt,
	numberstyle=\tiny\color{mygray},
	stringstyle=\color{mymauve},
	showstringspaces=false,
	showspaces=false,
	showtabs=false,
	tabsize=2,
	framexleftmargin=10mm,
	frame=none,
	backgroundcolor=\color[RGB]{245,245,244},
	keywordstyle=\bf\color{blue},
	identifierstyle=\bf,
	numberstyle=\color[RGB]{0,192,192},
	commentstyle=\it\color[RGB]{0,96,96},
	stringstyle=\rmfamily\slshape\color[RGB]{128,0,0}
}

\newcommand{\tabincell}[2]{\begin{tabular}{@{}#1@{}}#2\end{tabular}}

\usepackage{algorithm}
\usepackage{algorithmicx}
\usepackage{algpseudocode}


\begin{document}

\begin{titlepage}
\vspace*{40mm}
\begin{center}
{\Huge Homework \quad 6}\\[30mm]

{\Large 5130309059 \quad \quad 李佳骏}\\[3mm]
\texttt{taringlee@sjtu.edu.cn}\\[10mm]

2015.11.22

\end{center}
\end{titlepage}

\section*{Exercise 6.5}
Prove that for most Boolean function: $f : ({0,1}^n)^k \rightarrow {0,1}$, $D(f) = \Omega(n)$.

Use the hint: Count the number of protocols of a given cost.\\
At first, there are at most $2^{2^{nk}}$ Boolean function in the exercise. Now we count the number of different protocol trees with the deep less than $l$. There are $k2^{2^{(k-1)n}}$ cylinders and  the protocol tree's deep less than $l$ at most $2^l$ cylinders. So, we could calculate the upper bound $k^{2^l}2^{2^{(k-1)n}2^l} = k^{2^l}2^{2^{(k-1)n+l}}$ which exist one-to-one match between protocols and Boolean functions.\\
Hence we compare at $2^{2^{nk}}$ and $ k^{2^l}2^{2^{(k-1)n+l}}$, while $l = n - c$ (constant $c \geq 1$ is enough). the ratio of two values is exponential small enough.

\section*{Exercise 6.25}
Let $A$ be an $n$-bit string, and $1 \leq j,i \leq n$. \\
Define the 3-argument function $\mathrm{SUM-INDEX} (A,j,i) = A[j \oplus i]$, where $\oplus$ denotes bitwise $xor$. Prove that $D^{||}(\mathrm{SUM-INDEX})=\Omega(\sqrt{n})$.

Use the hint: Reduction from INDEX.\\
This exercise is a special case of \textbf{Example 6.22} where k = 3. Let $A_{\oplus}$ be a 2 dimensional array of bits, where each dimension has n entries from string $A$, and satisfied the equation $A_{\oplus}[i_1,i_2] = A[i_1] \oplus A[i_2]$. For every $j$ $(1 \leq j \leq 2)$, let $i_j$ be an integer $1 \leq i_j \leq n$. Thus $A_{\oplus}$ is represented by $N = n^2$ bits and each $i_j$ by $\log{n}$ bits. The function $\mathrm{SUM-INDEX(i_1,i_2,A_{\oplus})}$ (replace $i,j$ to $i_1,i_2$) is defined to be the $(i_1,i_2)$-th entry of $A_{\oplus}$, that is $A_{\oplus}[i_1,i_2]$.\\
Hence SUM-INDEX problem is reduced to INDEX problem. and $D^{||}(\mathrm{SUM-INDEX})=\Omega(\sqrt{n})$.

\end{document}