\documentclass[11pt, fleqn, a4paper]{report}
%a4paper : 21.0cm * 29.7cm

\usepackage{xeCJK}
\setCJKmainfont[BoldFont=STFangsong, ItalicFont=STKaiti]{STSong}
\setCJKsansfont[BoldFont=STHeiti]{STSong}
\setCJKmonofont{STFangsong}

\usepackage{amsmath}
\usepackage{amssymb,amsfonts}
\usepackage{tabularx}
%\usepackage{longtable}
\usepackage{graphicx}
\usepackage{multirow}
\usepackage{tikz}
\usepackage[T1]{fontenc}
\usepackage{upquote}
\usepackage[colorlinks, linkcolor=blue, anchorcolor=blue, citecolor=blue, urlcolor=blue]{hyperref}
\usepackage{ltxtable, filecontents}
\usepackage{mathrsfs}  

\setlength{\topmargin}{0cm}

\setlength{\oddsidemargin}{0.4cm}
\setlength{\evensidemargin}{0.4cm}
\setlength{\hoffset}{-0.2in}
\setlength{\textwidth}{440pt}
\setlength{\textheight}{650pt}
\setlength{\parindent}{0pt}
\setlength{\parskip}{5pt}

\setlength{\mathindent}{0pt}

\usepackage{color}
\usepackage{xcolor}
\usepackage{listings}

\usepackage{caption}
\DeclareCaptionFont{white}{\color{white}}
\DeclareCaptionFormat{listing}{\colorbox{lightgray}{\parbox{\textwidth}{#1#2#3}}}
\captionsetup[lstlisting]{format=listing,labelfont=white,textfont=white}

\definecolor{mygreen}{rgb}{0,0.6,0}
\definecolor{mygray}{rgb}{0.5,0.5,0.5}
\definecolor{mymauve}{rgb}{0.58,0,0.82}

\lstset{
	basicstyle=\footnotesize,
	breaklines=true,
	commentstyle=\color{mygreen},
	numbers=left,
	numbersep=5pt,
	numberstyle=\tiny\color{mygray},
	stringstyle=\color{mymauve},
	showstringspaces=false,
	showspaces=false,
	showtabs=false,
	tabsize=2,
	framexleftmargin=10mm,
	frame=none,
	backgroundcolor=\color[RGB]{245,245,244},
	keywordstyle=\bf\color{blue},
	identifierstyle=\bf,
	numberstyle=\color[RGB]{0,192,192},
	commentstyle=\it\color[RGB]{0,96,96},
	stringstyle=\rmfamily\slshape\color[RGB]{128,0,0}
}

\newcommand{\tabincell}[2]{\begin{tabular}{@{}#1@{}}#2\end{tabular}}

\usepackage{algorithm}
\usepackage{algorithmicx}
\usepackage{algpseudocode}


\begin{document}

\begin{titlepage}
\vspace*{40mm}
\begin{center}
{\Huge Homework \quad 5}\\[30mm]

{\Large 5130309059 \quad \quad 李佳骏}\\[3mm]
\texttt{taringlee@sjtu.edu.cn}\\[10mm]

2015.11.16

\end{center}
\end{titlepage}

\section*{Exercise 5.11}
Prove that $C^D(R) \geq n^2$ is the best lower that can be proven by using \textbf{Lemma 5.9}

To use \textbf{Lemma 5.9} for the relation $R_{\oplus}$, we take X to be the set of all strings whose parity is 1 and Y to be the set of all strings whose parity is 0. In this case, the relation R defined in Lemma 5.9 is exactly $R_{plus}$. In addition, note that $|X| = |Y| = 2^{n-1}$, whereas $|C| = n2^{n-1}$(because for every $x \in X$ each of the $n$ strings in Hamming distance $1$ from $x$ is in $Y$). Hence, $C^{D}(R_{\oplus}) \geq n^2$, which implies $D_{R_{\oplus}} \geq 2 \log{n}$. Now we show that $C_{D}(R) \geq n^2$ is a lower bound can be proven by using \textbf{Lemma 5.9}.

Thus, we will show that it's the best lower bound. By definition $$C = \{(x,y) : x \in X, y \in Y, d(x,y) = 1\}$$ We notice that there are at most $n$ elements in $Y$ with fixed $x \in X$, which means the size of $C \leq \min(n|X|, n|Y|)$. Without loss of generality, assume that $|Y|$ is always less than $|X|$, set $C \leq n|Y|$.

So, the inequality $C^D(R) \geq \frac{|C|^2}{|X||Y|} \geq \frac{n^2|Y|}{|X|} = n^2\frac{|Y|}{|X|} \geq n^2$.

\section*{Exercise 5.21}
Let FORK$'$ be the relation consisting of all triples $(x,y,i)$ such that $x,y \in \sum^{l}$ and $i$ is such that $x_i = y_i$ and either $x_{i+1} \neq y_{i+1}$ or $x_{i-1} \neq y_{i-1}$. Prove that $D(\mathrm{FORK}') = \Omega(\log{l}\log{w})$

We will follow the prove of \textbf{Corollary 5.20} to proof it.

At first, let's consider that If there exists a c-bit $(\alpha, l)$ protocol for the relation FORK$'$, then there is also a $c-1$-bit $(\frac{\alpha}{2}, l)$ protocol for FORK$'$. The prove is like \textbf{Lemma 5.17} which means this lemma can suit at FORK$'$.

And we prove the lemma like $\textbf{Lemma 5.18}$, let $\alpha \geq \frac{\lambda}{w}$ for a large enough constant $\lambda$. If there exists a $c$-bit $(\alpha, l)$ photocol for FORK$'$, then there is also a $c$-bit $(\frac{\sqrt{\alpha}}{2}, \frac{l}{2})$ protocol for it. There is a row, corresponding to some string $u$, whose density is at least $\sqrt{\frac{\alpha}{2}}$. The new protocol works as follows, Alice and Bob use the original c-bit protocol on the lenght-$l$ strings $ux$ and $uy$(and subtract $\frac{l}{2}$ from the output). Because the same string $u$ is concatenated to both $x$ and $y$, then the output of the protocol is guaranteed to be in the second half of the string. The protocol is a $(\frac{\sqrt{\alpha}}{2}, \frac{l}{2})$. hence there are $\sqrt{\frac{\alpha}{2}}n$row whose density at least $\frac{\alpha}{2}$, we just control that there are more than $\frac{\sqrt{\alpha}}{2}$inputs and to be outputs directly.

Finally, we can prove $D(\mathrm{FORK}') = \Omega(\log{l}\log{w})$ easily to use lemma above like what we did in \textbf{Corollary 5.20}. Clearly, $c(1,l) \geq c(\frac{1}{w^{1/3}, l}) \geq \Omega(\log{w})+c(\frac{1}{w^{2/3}}, l)  \geq \Omega(\log{w})+c(\frac{1}{w^{1/3}}, l/2)$ and inductively $\theta(\log{l})$ times, we prove it.

\end{document}