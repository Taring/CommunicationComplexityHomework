\documentclass[13pt, fleqn, a4paper]{report}
%a4paper : 21.0cm * 29.7cm

\usepackage{xeCJK}
\setCJKmainfont[BoldFont=STFangsong, ItalicFont=STKaiti]{STSong}
\setCJKsansfont[BoldFont=STHeiti]{STSong}
\setCJKmonofont{STFangsong}

\usepackage{amsmath}
\usepackage{amssymb,amsfonts}
\usepackage{tabularx}
%\usepackage{longtable}
\usepackage{graphicx}
\usepackage{multirow}
\usepackage{tikz}
\usepackage[T1]{fontenc}
\usepackage{upquote}
\usepackage[colorlinks, linkcolor=blue, anchorcolor=blue, citecolor=blue, urlcolor=blue]{hyperref}
\usepackage{ltxtable, filecontents}

\setlength{\topmargin}{0cm}

\setlength{\oddsidemargin}{0.4cm}
\setlength{\evensidemargin}{0.4cm}
\setlength{\hoffset}{-0.2in}
\setlength{\textwidth}{440pt}
\setlength{\textheight}{650pt}
\setlength{\parindent}{0pt}
\setlength{\parskip}{5pt}

\setlength{\mathindent}{0pt}

\usepackage{color}
\usepackage{xcolor}
\usepackage{listings}

\usepackage{caption}
\DeclareCaptionFont{white}{\color{white}}
\DeclareCaptionFormat{listing}{\colorbox{lightgray}{\parbox{\textwidth}{#1#2#3}}}
\captionsetup[lstlisting]{format=listing,labelfont=white,textfont=white}

\definecolor{mygreen}{rgb}{0,0.6,0}
\definecolor{mygray}{rgb}{0.5,0.5,0.5}
\definecolor{mymauve}{rgb}{0.58,0,0.82}

\lstset{
	basicstyle=\footnotesize,
	breaklines=true,
	commentstyle=\color{mygreen},
	numbers=left,
	numbersep=5pt,
	numberstyle=\tiny\color{mygray},
	stringstyle=\color{mymauve},
	showstringspaces=false,
	showspaces=false,
	showtabs=false,
	tabsize=2,
	framexleftmargin=10mm,
	frame=none,
	backgroundcolor=\color[RGB]{245,245,244},
	keywordstyle=\bf\color{blue},
	identifierstyle=\bf,
	numberstyle=\color[RGB]{0,192,192},
	commentstyle=\it\color[RGB]{0,96,96},
	stringstyle=\rmfamily\slshape\color[RGB]{128,0,0}
}

\newcommand{\tabincell}[2]{\begin{tabular}{@{}#1@{}}#2\end{tabular}}

\usepackage{algorithm}
\usepackage{algorithmicx}
\usepackage{algpseudocode}


\begin{document}

\begin{titlepage}
\vspace*{40mm}
\begin{center}
{\Huge Homework \quad 2}\\[30mm]

{\Large 5130309059 \quad \quad 李佳骏}\\[3mm]
\texttt{taringlee@sjtu.edu.cn}\\[10mm]

2015.10.8

\end{center}
\end{titlepage}

\section*{Theorem 2.11(COMBINATORIAL VERSION)}
Prove that for every function $f : X \times Y \rightarrow \{0,1\}$, $$D(f) = O(N^0(f)N^1(f))$$

首先我们制定一些定义和规定。
\begin{itemize}
\item 定义$F_B$表示所有布尔函数的集合。
\item 定义一个二元函数$L(k,l)$, 满足:$$L(k,l) = \max_{g \in F_B}^{C^1(g) \leq k, C^0(g) \leq l}{C^P(g)}$$
\end{itemize}
考虑布尔函数$f$的最优化覆盖。令$R = S \times T$为其中的一个0单色矩形阵(rectangle),$R' = S' \times T'$为其中的一个1单色矩形阵。由于$R \cap R' = \varnothing$,所以$S \cap S' = \varnothing$或$T \cap T' = \varnothing$至少有一个成立。不失一般性的,令至少有一半的$R'$存在且满足$S \cap S' = \varnothing$;否则对矩阵$M_f$做转置即可。

现在构造一个合适的协议(protocol)。
\begin{itemize}
\item 首先,Alice向Bob传达信息$x \in S$是否成立。
\item 若成立,则空间可缩小到$S \times Y$,而这样的话至多只有$\frac{k}{2}$个1单色矩形阵。
\item 若不成立,则空间可缩小到$\overline{S} \times Y$,而这样的话至多只有$l - 1$个0单色矩形阵。
\end{itemize}
由此,可得公式$$L(k,l) \leq L(\frac{k}{2}, l) + L(k, l - 1) \leq (l + 1)^{\log{k}}$$

由定理2.8,$D(f) = O(\log{C^P( f ) })$,可得$$D(f) = O(\log{(L(C^1(f),C^0(f)))}) = O(\log{(C^0(f) + 1)^{\log{C^1(f)}}}) = O(\log{C^0(f)}\log{C^1(f)})$$

即,$$D(f) = O(N^0(f)N^1(f))$$

\section*{Exercise 2.13}
Let $f$ be any function for which $X \times Y$ can be covered by t f-monochromatic geometric rectangles(possibly with overlaps). Prove that $D(f) = O(\log{t})$.

对于任意一种覆盖方案,都可以直接构建一个协议,则满足$C^P(f) \leq t$;显然的,$t \leq 2^{D(f)}$。所以得,$$C^P(f) \leq t \leq 2^{D(f)}$$

这样的协议(protocol)构建有很多种方法,以下举证其中的一种:
\begin{itemize}
\item Alice和Bob预处理,将rectangle进行离散化处理。
\item Alice把x所在的区域的编号传个Bob
\item Bob综合信息,找到$M_{xy}$,输出结果
\end{itemize}

根据定理2.8,可知$D(f) = O(\log{C^P(f)})$。由广义夹逼定理可得,$D(f) = O(\log{t})$。

\section*{Exercise 2.18}
Show that most functions $f : \{0,1\}^n \times \{0,1\}^n \rightarrow \{0,1\} \;$satisfy $N^1(f) = \Omega(n)$, and yet the size of the largest fooling set for $f$ is $O(n)$.

考虑一个随机布尔函数$f_r : \{0, 1\}^n \times \{0, 1\}^n \rightarrow \{0, 1\}$。其中对于任意的$(x,y)$与$(x',y')$,满足$x,x' \in X$,$y,y' \in Y$,$f(x,y)$与$f(x',y')$的取值相互独立,且为0或1的概率相等。

因此,矩阵$M_{f_r}$的元素为1的个数的期望是$2^n*2^n / 2= 2^{2n - 1}$. 由Chernoff不等式,1的个数小于$2^{2n - 2}$的概率是$\epsilon_1 = 2e^{-\frac{n}{8}}$. 

而若$N^1(f_r) = \Omega(n)$,则有极大的概率不存在$8*2^n = 2^{n + 3}$个矩形阵(rectangle),存在单个这样的rectangle的概率是$2^{-2^{(n+3)}}$。则整体存在这样一个矩形阵的概率为$\epsilon_2 = 2^{2^n}2^{2^n}2^{-2^{(n+3)}} = 2^{-6 \cdot 2^{n}}$.

在不存在这样的矩形下继续讨论。而若"the size of the largest fooling set for $f$ is $O(n)$"成立。则可假设不存在大小为$8n$的fooling set。
出现这样的一个fooling set的可能性为$(\frac{3}{4})^{C(8n, 2)} < 2^{-80 n^2}$,而出现的次数为$C(2^{2n}, 8n) < 2^{16 n^2}$. 所以存在$8n$的fooling set 的概率为$\epsilon_3 = 2^{-80 n^2} * 2^{16 n^2} = 2^{-64 n^2}$

由此,with high probability$(1 - \epsilon_1 -\epsilon_2 - \epsilon_3)$,一个随机函数$f$满足$N^1(f) = Omega(n)$且最大的fooling set不超过$O(n)$

\section*{Exercise 2.23}
\subsection*{Subproblem 1}
Show that the rank of $M_{\mathrm{INTER}}$ is n.

不难发现INTER作为点积,和矩阵乘法类似。所以可以构造$2^n \times n$的01矩阵A,其中该矩阵每一行表示X的一个元素,共$2^n$个。事实上由于X集合和Y集合的元素相同,可得$M_{\mathrm{INTER}} = A A^T$。显然的,矩阵A的秩为n。

由线性代数定理可知:$$rank(A) + rank(A^T) - n \leq rank(AA^T)  = rank(M_{\mathrm{INTER}}) \leq \min(rank(A),rank(A^T))$$

解得,$ rank(M_{\mathrm{INTER}})  = n$.

显然的,$D(f) = n + 1$。所以,$D(f)$ and $\log{rank(f)}$ may be exponential.

\subsection*{Subproblem 2}
Show that the rank of $M_{\mathrm{IP}}$ over $GF(2)$ is n.

不难发现INTER和IP这两个操作极为相似。做类似的代换即可获得答案,即$ rank(M_{\mathrm{IP}})  = n$。

当然,类似的可以求得$D(f) = \Omega(n)$。所以,$D(f)$ and $\log{rank(f)}$ may be exponential。

\end{document}